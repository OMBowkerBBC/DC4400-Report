\section{Introduction}

  The BBC is now over 100 years old [1] and is well known for it's TV channels and radio stations that are broadcast over the airwaves [2] 
  to peoples homes across the UK. However this old way of broadcasting, sending out airwaves on a certain frequency to an antenna, is becoming 
  less popular in the modern age of the internet. A study done by Ofcom [3] showed that people
  \textit{'watched on average about 16\% less broadcast TV between 2019 ... and 2022'} [4], with viewing \textit{'decreasing by 47\%'} [4] between ages
  16-24. In addition to this another study carried out by media analyst firm Ampere found that in 2021 37\% of people claimed to watch no linear TV,
  this increased to 45\% by 2023 [5].
  
  This fall correlates with the significant rise in internet enabled TVs in the home, with statista finding that 
  \textit{'In 2014 just 11 percent of households in the UK owned a Smart TV, whereas, in 2023, nearly 74 percent of households reported owning a Smart TV.'} [6].
  Some of these devices still support OTA broadcasts, however devices like the Amazon Fire TV stick and Googles Chromecast, are purely internet
  based; However they do offer a \textit{'guide/epg'} section with Amazon having a development guide [7] on how to integrate with it.
  Director general of the BBC, Tim Davie, in a 2022 stated \textit{'The vision is simple: from today we are going to move decisively to 
  a digital-first BBC'} [8], giving more priority on these new forms of media. 
  
  This report will discuss an upgrade carried out to the BBCs \textit{'off-product'} schedules system, responsible for delivering up to date schedules to
  companies such as Freeview, Amazon and more. First I will give some background on the project, this will include research done, the starting
  architecture of the system and how it aligns with the BBCs and teams OKRs [9]. Following that, I will discuss the work that was done, this will 
  include the planning, writing, testing and deployment stages of the project, discussing findings along the way. Then a short section on the final product 
  that has been produced followed by a final discussion of future upgrades to the system that could be carried out.

\newpage
